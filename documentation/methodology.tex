\section{Methodology}
The objective of the project was to produce a parameterized simulation 
that could represent an ad-hoc networking scenario and compare the effects 
of different parameters on the overall communications system in terms of 
per-user average success rate (succssfully received messages) 
and latency of those messages.

The simulation was developed in several compontents.
The first stage of the simulation is the map creation layer,
which places N users (ceullar devices) randomly on an \textit{n x m} map at a
given (x, y) coordinate with a given starting radius (communication range).
The stage is able to instaniated for a new simulation dynamically, or additional
methods exist to process files containing generated lists of users so that
statically determined arrangements can be tried.
This layer was leveraged to create several static lists of different quantity
and arrangements of users, which were used for analysis by the main simulation.

The second stage was the main simulation.
This stage took a list of users, as described previously, as well as additional
parameters describing the nature of communication amoung the users such as
the intensity ratio and the number of steps (as an analogue for time) to run 
the simulation for.
At each stage of the simulation, a randomly chosen subset of all users on the map
were chosen based upon the intensity ratio to attempt to send a message to another
randomly chosen user on the map.
After the number of steps specified was reached, the simulation would continue
to allow all users to complete their retransmission queues, which would usually
take a few extra steps to complete.
Once all users had finished their retransmissions, statistics would be compiled
on a per-user basis, and overall averages would be returned.

The main simulation stage was composed of several smaller elements modeling
the messages and user interactions.
These elements were modeled relatively simplisitically, so future additions could be
made to model other complex aspects of communication e.g. multi-packet messages,
fluctuating communication range, power usage, etc.

Finally, the last stage was a statistical analysis layer which enabled taking overall
statistics over a range of parameter values and creating relevant tables and graphs,
which are used in the results portion of this paper.

The language chosen to implement the simulation was Python, due to it's incorporation of 
object-oriented paradigms, as well as the ease of prototyping a larger program in a 
short period of time.
Additionally, the author's knowledge and existance of several data tools made it a natural
fit for this project.
A full listing of all the code created for this simulation is in Appendix \ref{appendix:code}.
