\section{Results and Discussion}
%TODO: Complete
The simulation was run for independent variations of four parameters under constant conditions.
The constant conditions for the experiment are described in Figure \ref{fig:consttable} below.

\begin{figure}[!htb]
    \centering
    \begin{tabular}{c|rl}
        \hline
        Parameter               & Value     & Units         \\
        \hline
        Grid Area               & 100x100   & feet squared  \\
        Transmission Radius     & 10        & feet          \\
        User Population         & 200       & users         \\
        \hline
        Intensity Ratio         & 0.05      & none          \\
        Simulation Duration     & 50        & steps         \\
        \hline
    \end{tabular}
    \caption{Table of constants}
    \label{fig:consttable}
\end{figure}

\subsection{User Area Separation Variation}
Figure \ref{fig:vararea} shows the results when the area is increased
(also increasing average user separation).

\sidebysidefigures
{vararea-success-rate-graph.tex}       {Success Rate vs. Area}
{vararea-average-latency-graph.tex}    {Average Latency vs. Area}
{Network Properties with Increasing Area given Constant Radius and Number of Users}
{fig:vararea}

Figure \ref{fig:vararea}a shows a exponential decay in successfully received packets.
The decay starts from 100\% corresponding to the point where the users are crowded very close
together (user transmission radius approximately equal to grid size) and swiftly decays as
the grid size is increased.
This result means that as separation distance increases for a small population, there
is a corresponding drop-off in performance that quickly makes this kind of network non-viable.
In practice, this would mean that connecting users would need to increase their transmission radius
and therefore their power in order to overcome this limitation.
However, as cellular devices are small, this solution has limited effect.

Figure \ref{fig:vararea}b shows a logarithmic increase in latency.
The results become very erratic when the area is very large because so few messages
are received at that point.
The effect tails off as user separation increases because, even though the probability
of success is greatly reduced, the successful paths that can be formed between users
occur when communication is traveling between well-placed users in the communication path.
This increase in area shares the largest possible latencies overall as when a message does
successful travel to it's intended recipient, it has been re-transmitted by participating
users many times.
There is an artificial limit to this latency in the fact that each message has a Time-To-Live (TTL)
parameter which controls how many hops it can make before it is discarded.
In areas where user separation far outpaces transmission radius, the intuition here is to
increase the TTL message parameter to ensure a greater success rate.

\subsection{User Transmission Radius Variation}
Figure \ref{fig:varradius} shows the results when the user transmission radius is increased.

\sidebysidefigures
{varradius-success-rate-graph.tex}     {Success Rate vs. Radius}
{varradius-average-latency-graph.tex}  {Average Latency vs. Radius}
{Network Properties with Increasing Radius Area given Constant Area and Number of Users}
{fig:varradius}

Figure \ref{fig:varradius}a shows a logarithmic increase in successfully received packets.
The success rate increases from 0\% near the trivial case when user radius is much less
than the grid size and user population can support, to 100\% again near the region where
the transmission radius approximately equals grid size.
The intuition here mirrors the intuition for the increasing grid size and produces a strong
correlation towards increasing the transmission radius (at the expense of cellular battery life)
when the user population and separation distance do not support normal communication.

Figure \ref{fig:varradius}b shows a exponential decay in in latency.
There is an asymptote that is reached near the point where user transmission radius is
close to the same region where separation distance and transmission radius are approximately
equal. This asymptote could appear due to interference between transmissions as the radius 
gets larger. The intuition here is that there is an upper limit to the effectiveness of increasing
the user transmission radius based on the population size and separation distance of the users.

\subsection{User Population Variation}
Figure \ref{fig:varusers} shows the results when the user population is increased.

\sidebysidefigures
{varusers-success-rate-graph.tex}      {Success Rate vs. Number of Users}
{varusers-average-latency-graph.tex}   {Average Latency vs. Number of Users}
{Network Properties with Increasing Number of Users given Constant Area and Radius}
{fig:varusers}

Figure \ref{fig:varusers}a shows a steep exponential decrease in successfully received packets.

Figure \ref{fig:varusers}b shows a slow but exponential decrease in latency with an asymptote.

\subsection{Messaging Intensity Ratio Variation}
Figure \ref{fig:varintensity} shows the results when the transmission intensity ratio is increased.
Note this is the only parameter that this parameter was the only parameter modeled in simulation
based on the overall user population, instead of characteristics of the individual users.

\sidebysidefigures
{varintensity-success-rate-graph.tex}      {Success Rate vs. Intensity Ratio}
{varintensity-average-latency-graph.tex}   {Average Latency vs. Intensity Ratio}
{Network Properties with Increasing Intensity Ratio given Constant Area, Radius and Users}
{fig:varintensity}

Figure \ref{fig:varintensity}a shows a steep exponential decrease in successfully received packets.

Figure \ref{fig:varintensity}b shows a slow but exponential decrease in latency with an asymptote.
