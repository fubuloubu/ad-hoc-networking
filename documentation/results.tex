\section{Results and Discussion}
The simulation was run for independent variations of four parameters under constant conditions.
The constant conditions used as the baseline for the experiment are described in 
Figure \ref{fig:consttable}.
The ranges over which all the parameters were individually varied are listed in 
Figure \ref{fig:rangetable}.

\begin{figure}[!htb]
    \centering
    \begin{tabular}{l|rl}
        \hline
        Parameter               & Value     & Units                     \\
        \hline
        Grid Area               & 100x100   & feet squared              \\
        Transmission Radius     & 10        & feet                      \\
        User Population         & 200       & users                     \\
        \hline
        Intensity Ratio         & 0.05      & msgs per user per step    \\
        Simulation Duration     & 50        & steps                     \\
        \hline
    \end{tabular}
    \caption{Table of constants}
    \label{fig:consttable}
\end{figure}

\begin{figure}[!htb]
    \centering
    \begin{tabular}{l|rr}
        \hline
        Parameter               & Min   & Max       \\
        \hline
        Grid Area               & 10x10 & 160x160   \\
        Transmission Radius     & 10    & 160       \\
        User Population         & 100   & 1600      \\
        \hline
        Intensity Ratio         & 0.010 & 0.100     \\
        \hline
    \end{tabular}
    \caption{Parameter Ranges}
    \label{fig:rangetable}
\end{figure}

\subsection{User Area Separation Variation}
Figure \ref{fig:vararea} shows the results when the area is increased
(also increasing average user separation).

\sidebysidefigures
{vararea-success-rate-graph.tex}       {Success Rate vs. Area}
{vararea-average-latency-graph.tex}    {Average Latency vs. Area}
{Network Properties with Increasing Area given Constant Radius and Number of Users}
{fig:vararea}

Figure \ref{fig:vararea}a shows a exponential decay in successfully received packets.
The decay starts from near 100\% corresponding to the point where the users are crowded very close
together (user transmission radius approximately equal to grid size) and swiftly decays as
the grid size is increased.
This result means that as separation distance increases for a small population, there
is a corresponding drop-off in performance that quickly makes this kind of network non-viable.
In practice, this would mean that connecting users would need to increase their transmission radius
and therefore their power in order to overcome this limitation.
However, as cellular devices are small, this solution has limited effect.

Figure \ref{fig:vararea}b shows a logarithmic increase in latency.
The results become very erratic when the area is very large because so few messages
are received at that point.
The effect tails off as user separation increases because, even though the probability
of success is greatly reduced, the successful paths that can be formed between users
occur when communication is traveling between closely placed users in the communication path.
This increase in area shares the largest possible latencies overall as when a message does
successful travel to it's intended recipient, it has been re-transmitted by participating
users many times.
There is an artificial limit to this latency in the fact that each message has a Time-To-Live (TTL)
parameter which controls how many hops it can make before it is discarded.
In areas where user separation far outpaces transmission radius, the intuition here is to
increase the TTL message parameter to ensure a greater success rate.

\subsection{User Transmission Radius Variation}
Figure \ref{fig:varradius} shows the results when the user transmission radius is increased.

\sidebysidefigures
{varradius-success-rate-graph.tex}     {Success Rate vs. Radius}
{varradius-average-latency-graph.tex}  {Average Latency vs. Radius}
{Network Properties with Increasing Radius Area given Constant Area and Number of Users}
{fig:varradius}

Figure \ref{fig:varradius}a shows a logarithmic increase in successfully received packets.
The success rate increases from close to 0\% near the trivial case when user radius is far smaller
than the user density can support, to complete success again past the region where
the transmission radius approximately equals grid size (optimal user density).
The intuition here mirrors the intuition for the increasing grid size and produces a strong
correlation towards increasing the transmission radius (at the expense of cellular battery life)
when the user density does not support normal communication.

Figure \ref{fig:varradius}b shows a exponential decay in in latency.
There is an asymptote that is reached near the point where user transmission radius is
close to the same region where separation distance and transmission radius are approximately
equal. This asymptote could appear due to interference between transmissions as the radius 
gets larger. The intuition here is that there is an upper limit to the effectiveness of increasing
the user transmission radius based on the population size and separation distance of the users.

\subsection{User Population Variation}
Figure \ref{fig:varusers} shows the results when the user population is increased.

\sidebysidefigures
{varusers-success-rate-graph.tex}      {Success Rate vs. Number of Users}
{varusers-average-latency-graph.tex}   {Average Latency vs. Number of Users}
{Network Properties with Increasing Number of Users given Constant Area and Radius}
{fig:varusers}

Figure \ref{fig:varusers}a shows an exponential decrease in successfully received packets with
an asymptote. The asymptote appears to start around 500 users, and experiences a much gradual
linear decay after that. The point where the asymptote starts corresponds to the same saturation 
point in user density as mentioned prior. The intuition here would point to
a ceiling on the capacity that such a system can support. This could be an artificial limit based
on the size of the re-transmit stack allocated to each user device as well as the TTL of each message.
If the stack where increased, more messages could be re-transmitted potentially increasing the
likelihood of success. However, as the number of users increase the amount of interference and
capacity considerations might limit the effectiveness of such a solution. Increasing the TTL
parameter would be a better solution to increasing the success rate since a message would
logically need more hops to make it to it's destination when the area is more crowded.
This solution would also increase the overall amount of transmissions, so a trade would need to be
made in order to avoid reduced returns.

Figure \ref{fig:varusers}b shows a exponential decrease in latency. The intuition here, when coupled
with the success rate, suggests that the successful messages are between two users that are 
increasingly closer to each other as the space gets more crowded. Increasing the TTL to gain a higher
success rate would have the effect of increasing the average latency of successful messages, as
again more hops would be needed to reach destinations as the user density increases.

\subsection{Messaging Intensity Ratio Variation}
Figure \ref{fig:varintensity} shows the results when the transmission intensity ratio is increased.
Note this is the only parameter that was modeled in simulation based on the overall user population, 
instead of characteristics about the individual users.

\sidebysidefigures
{varintensity-success-rate-graph.tex}      {Success Rate vs. Intensity Ratio}
{varintensity-average-latency-graph.tex}   {Average Latency vs. Intensity Ratio}
{Network Properties with Increasing Intensity Ratio given Constant Area, Radius and Users}
{fig:varintensity}

Figure \ref{fig:varintensity}a shows a semi-linear decrease in successfully received packets.
The increase in intensity corresponds to a sharp and then gradual decrease in the rate of success.
The initial sharp decrease corresponds to the area before saturating the same threshold of
user density. However, the intuition
for the more gradual decrease after that point suggests that as the intensity increases, there
is a higher likelihood that the messaging is occurring between users that are closer together.
This corresponds to figure \ref{fig:varintensity}b, which shows a steep linear decrease in latency.
Reviewing these two graphs in parallel suggests this intuition is correct, in that an increase
in the amount of communication overall increases the load on the system, and reduces the effective
range of messaging between two users. Similar to the user density increase, the solution here
would be to increase the TTL of the messaging. However, with this increase in the liveness of
individual messages coupled with the overall increase in the number of messages, there may exist
a structural limit in the capacity of this system to deliver messages even with the increased 
liveness due because of interference.
