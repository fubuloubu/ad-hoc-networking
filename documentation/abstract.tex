Wireless Ad-Hoc Networks are a very active area of current research.
They have the ability to supplement or completely supplant wireless connectivity between peer devices,
especially in circumstances like protests, concerts, and other outdoor events where a large number of
unexpected wireless users need service, overwhelming traditional base stations.
This technology will allow communication between two participants irregardless of the load on the
traditional systems, and can dynamically grow and adapt to changing circumstances in order to provide
a robust conduit for communication.
However, this technology also has a myriad of issues due to managing interference, and is particular
sensitive to the number of users available: too little, and the network has a hard time making
connections, too many and the interference makes routing very difficult.
This project seeks to model the communication properties of mesh networks, 
measure their effectiveness (latency and dropped packets), 
and provide suggestions for improvements by simulating over a wide range of end users, 
modeled using the assumption that all devices are communicating using a simplified protocol
using homogenous hardware.
