\section{Abstract}
Wireless Ad-Hoc Networks are a very active area of current research.
They have the ability to supplement or completely supplant wireless connectivity between peer devices,
especially in circumstances like protests, concerts, and other outdoor events where a large number of
unexpected wireless users need service, overwhelming traditional base stations.
This technology will allow communication between two participants irregardless of the load on the
traditional systems, and can dynamically grow and adapt to changing circumstances in order to provide
a robust conduit for communication.

However, this technology also has a myriad of issues due to managing interference,
and is particularly sensitive to the number of users available: 
too little, and the network has a hard time making connections;
too many, and the interference makes routing very difficult.
This project seeks to model the communication properties of Homogeneous Ad-Hoc networks, 
measure their effectiveness (dropped packets and messaging latency),
and provide a basis for drawing conclusions about such networks 
by varying different parameters governing their behavior, 
modeled using the assumption that all devices are communicating 
using a simplified On-Demand Routing protocol implemented on homogeneous hardware.
