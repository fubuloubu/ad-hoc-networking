\section{Conclusion}
In conclusion, we have summarized in our results some intuition to the behavior of Ad-Hoc Networks
over a variety of different situations.
We have described what the overall trend is in terms of latency and rate of success
over varying area, transmission radius, and user population, as well as messaging intensity.
All of this analysis was done under the assumptions of a re-transmission algorithm
utilizing a fixed-size stack and a messaging strategy that used a static TTL parameter
which allowed for a simplified model of the communication strategies these devices typically use.
Using the intuition we have built, we can make overall recommendations for algorithms that
actively manage parameters such as re-transmission stack size, user transmission radius/power
(and consequently user battery life), and TTL of the messages such that the likelihood of 
success is increased and the overall level of latency is maintained to a reasonable level.

While this simulation is a good baseline for categorizing Ad-Hoc Networks, the project could
benefit from a more realistic modeling of the style of communication, including more sophisticated
models for message size, messaging dynamics, power usage, and interference.
As it stands, this simulation may be an overly optimistic estimate of real system behavior and 
additional modeling may yield large differences in the results measured here.
However, this simulation provides a good backbone for those activities and hopefully future
projects may explore different possibilities in order to maximize per-user performance of
an Ad-Hoc messaging system.
