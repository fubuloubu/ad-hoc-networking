\section{Conclusion}
%TODO: work on this

In conclusion, we have summarized in our results some intuition to the behavior of Ad-Hoc Networks
over a variety of different situations.
We have described what the overall trend is in latency and successful transmission rate
over varying area, transmission radius, and user population, as well as transmission intensity.
All of this analysis was done under the assumptions of a simplified re-transmission algorithm
utilizing a queue that allowed for modeling the communication strategies these devices typically
use.

%Future work
While this simulation is a good baseline for categorizing Ad-Hoc Networks, the project could
benefit from more realistic modeling of the style of communication, including more sophisticated
models for message size, messaging dynamics, power usage, and interference.
As it stands, this simulation may be an overly liberal estimate of system behavior and additional
modeled may yield to large differences in the parameters measured here.
Also, different re-transmission algorithms can be modeled in future projects to compare
the effectiveness of different schemes.
However, this simulation provides a good backbone for those activities and hopefully future
projects may explore these possibilities.
